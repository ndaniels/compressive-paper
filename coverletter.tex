\documentclass{article}
\usepackage{color}
\usepackage{soul}
\usepackage{hyperref}
\hypersetup{
    colorlinks = true
}

\newcommand{\email}[1]{\href{mailto:#1}{#1} }

\setlength{\parindent}{0pt}
\setlength{\parskip}{1em}

\begin{document}
\hspace*{0.5\linewidth}

\today

\bigskip

Dear Dr. Mak,

We wish to submit the enclosed manuscript entitled,
``\textbf{Entropy-scaling search of massive biological data},''
by Y. William Yu, Noah M. Daniels, David C. Danko, and myself
to \textit{Cell Systems} as an Article.

One of the principal challenges facing biologists of every stripe is the task 
of organizing, searching, and analyzing vast amounts of data.
Because computing resources cannot keep pace with this explosion of data, in order to capitalize on the promise of Big Data, sophisticated algorithms and data structures are needed, as well as fast practical software implementations.

We \underline{meet this demand} by generalizing and formalizing compressive 
genomics: we demonstrate a general scheme for accelerating search in life science databases, on which we prove time and space complexity bounds in terms 
of entropy by exploiting the typically low fractal dimension of the underlying 
data. 
We also releasing three new practical software tools for practitioners to use.
 
To demonstrate the broad applicability of entropy-scaling search, we present acceleration of standard tools drawn from metagenomics, high-throughput drug screening, and protein structure search.
CaBLASTX achieves 700x speedup of BLASTX with $<5\%$ loss in sensitivity, Ammolite achieves 10x speedup of small molecule similarity search with $<1\%$ 
loss in sensitivity, and esFragBag achieves 10x speedup of FragBag with $<0.2\%$ loss in sensitivity.

Not only do these order-of-magnitude improvements in running time promise to enable new workflows for practitioners (e.g. local analyses of sequencing data in remote field sites for real-time epidemic monitoring, fast first-pass computational drug screens), but the general theory of entropy-scaling data structures that we introduce can be straightforwardly applied to accelerate other search tools.

It is for these reasons that we believe our work will become an important tool in the biologist's toolkit by allowing analyses to scale with with massive biological data, and is of broad interest to the readership of Cell Systems.

The manuscript has been seen and approved by all listed authors; no permissions or statements of personal communication are required.

The following are suggestions for referees:
\begin{itemize}
\item[] \textbf{Sorin Istrail}, Brown University, \email{sorin\_istrail@brown.edu}  %- theory
\item[] \textbf{Teresa Przytycka}, NIH, \email{przytyck@ncbi.nlm.nih.gov}  %-theory, molecule and fragment search
\item[] \textbf{Peter S. Kim}, Stanford University, \email{kimpeter@stanford.edu} %-- drugs
\item[] \textbf{Nir Ben-Tal}, Tel Aviv University, \email{bental@tauex.tau.ac.il} %- fragment search
\item[] \textbf{Michal Linial}, Hebrew University, \email{michal.linial@huji.ac.il}
\item[] \textbf{Martin Vingron}, Max Planck Institute, \email{martin.vingron@molgen.mpg.de}
\end{itemize}

May we request that due to \textbf{competing interests} the following individuals be excluded as referees:
\begin{itemize}
\item[] \textbf{Veli Makinen}, University of Helsinki
\item[] \textbf{Daniel Huson}, University of T\"ubingen
\item[] \textbf{Yuzhen Ye}, Indiana University
\end{itemize}


Sincerely,

Bonnie Berger  \\
Professor of Applied Mathematics and Computer Science, MIT  \\
Faculty Member, Harvard-MIT Health Science and Technology\\
Associate Member, Broad Institute of MIT and Harvard \\
Affiliated Faculty, Harvard Medical School

\end{document}
